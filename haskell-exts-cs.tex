%% haskell-exts-cs.tex
%
% Copyright 2018  Rudy Matela
%
% This text is available under (at your option):
%   * Creative Commons Attribution-ShareAlike 3.0 Licence
%   * GNU Free Documentation License version 1.3 or Later
%

\documentclass{refcard}
\usepackage[T1]{fontenc}
\usepackage{rotating}
\usepackage{amssymb}

\renewcommand{\familydefault}{\sfdefault}
\newcommand{\la}{\textbackslash}


\title{Haskell Extensions Cheat Sheet}

\cright{
	Copyright 2018-2019, Rudy Matela --
	Compiled on \today{} \\
	Upstream: \texttt{https://github.com/rudymatela/concise-cheat-sheets}
}{
	This text is available under
	the Creative Commons Attribution-ShareAlike 3.0 Licence, \\
	\textbf{or} (at your option), the GNU Free Documentation License version 1.3 or Later.
}
\version[~\\]{0.0}


\begin{document}

\maketitle

\section{Type Families (Type Functions)
  \hfill\C{\normalsize TypeFamilies}}

\begin{ldesc}
	\li[enabling] \C{\{-\# LANGUAGE TypeFamilies \#-\}}
\end{ldesc}\\[1em]
\begin{tabularlc}{r@{\s$\equiv$\s}l}
	type family          & type function \\	
	type family instance & type function pattern match \\
\end{tabularlc} \\[1em]
\begin{ldesc}
	\li[declaring]             type family \I{Family} a
	\li[instantiating]         type instance \I{Family} \I{TypeA} = \I{TypeB} \li

	\li[declaring (data)]      data family \I{DataFam} a
	\li[instantiating (data)]  data instance \I{DataFam} \I{Type} = \I{Conses...} \li

	\li[declaring (class)]
	class \I{Class} a where \li
	~~type \I{Family} a \li
	~~fun ::~...~->~\I{Family}~a~->~... \li

	\li[instantiating (class)]
	instance \I{Class} \I{TypeA} where \li
	~~type \I{Family} \I{TypeA}  =  \I{TypeB} \li
	~~fun ...~= ... \li
\end{ldesc}


\section{Tuple Sections
  \hfill\C{\normalsize TupleSections}}

\begin{ldesc}
	\li[enabling] \C{\{-\# LANGUAGE TupleSections \#-\}}
\end{ldesc}\\[1em]

\begin{tabularlc}{C@{\s$\equiv$\s}C}
	~~(x,)    & \la{}y -> (x,y) \\
	~~(,y)    & \la{}x -> (x,y) \\
	~(x,{},)  & \la{}y z -> (x,y,z) \\
	~(,y,)    & \la{}x z -> (x,y,z) \\
	~(,{},z)  & \la{}x y -> (x,y,z) \\
	(x,y,)   & \la{}z -> (x,y,z) \\
	(x,{},z) & \la{}y -> (x,y,z) \\
	(,y,z)   & \la{}x -> (x,y,z) \\
\end{tabularlc} \\[1em]

\begin{tabularlc}{C@{\s$\equiv$\s}C}
	~~~~~~~~~~~~~~~~(0,{},2) 1     &  (0,1,2) \\
	~~~~~~~~~~~(False,) True  &  (False,True) \\
	(False,{},0,{},{},) True 1 2 3 & (False,True,0,1,2,3) \\
\end{tabularlc} \\[1em]

\section{Lambda Case
  \hfill\C{\normalsize LambdaCase}}

\begin{ldesc}
	\li[enabling] \C{\{-\# LANGUAGE LambdaCase \#-\}}
\end{ldesc}\\[1ex]

\newcommand{\thead}[1]{\textnormal{\textbf{#1}}}
\newcommand{\theads}[2]{\multicolumn{1}{C@{\s$\approx$\s}}{\thead{#1}} & \thead{#2} \\}
\begin{tabular}{CC}
\theads{lambda case}{explicit lambda case}
\la{}case & \la{}\I{x} -> case~\I{x}~of \\
~~\I{pat1} -> \I{exA}~~~~ & ~~\T{x}~~~~~\I{pat1} -> \I{exA} \\
~~\I{pat1} -> \I{exA} & ~~\T{x}~~~~~\I{pat2} -> \I{exB} \\
~~\_\T{at2} -> \I{exC} & ~~\T{x}~~~~~\_\T{at2} -> \I{exC} \\[1ex]
\end{tabular}

\section{Generalized Algebraic Datatypes
  \hfill\C{\normalsize GADTs}}

\begin{ldesc}
	\li[enabling] \C{\{-\# LANGUAGE GADTs\#-\}}
\end{ldesc}\\[1ex]
\begin{tabular}{CC}
\theads{GADT}{regular data definition}
\\
data Maybe a where~~~~~~~~~~ & data~Maybe~a \\
~~Nothing~::~Maybe~a         & ~~=~Nothing  \\
~~Just~~~~::~a~->~Maybe~a    & ~~|~Just~a   \\
\\
data List a where                & data List a         \\
~~Nil~~::~List a                 & ~~= Nil             \\
~~Cons~::~a -> List a -> List a  & ~~| Cons a (List a) \\
\\
data [a] where                   & data [a] = [] | a:[a] \\
~~[]~~::~[a]                     & ~~\rotatebox{90}{$\Rsh$}\textnormal{\footnotesize ~(not real Haskell)} \\
~~(:)~::~a -> [a] -> [a]         & \\
\\
data Thing a where            & \textnormal{regular defn.~impossible} \\
~~Thing ::~a -> Thing a       & \\
~~AnInt ::~Int -> Thing Int   & \\
~~ABool ::~Bool -> Thing Bool & \\
\\
data Nat where      & data Nat = Z | S Nat \\
~~Z~::~Nat & \\
~~S~::~Nat -> Nat & \\
\\
data Vec n a where & \textnormal{regular defn.~impossible} \\
~~Nil~~~::~Vec Z a & \\
~~(:::)~::~a -> Vec n a -> Vec (S n) a \hspace{-8ex} & \\
\end{tabular}


\section{Rank-2 and Rank-N Types
  \hfill\C{\normalsize Rank2Types, RankNTypes}}

\end{document}
